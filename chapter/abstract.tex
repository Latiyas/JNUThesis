\frontmatter \pagenumbering{Roman}  %% 采用大写罗马数字,缺省时为小写的

\chapter*{摘\texorpdfstring{\qquad}{} 要}\addcontentsline{toc}{chapter}{摘要}
\thispagestyle{main}
%% 加星号*的章节默认是不加入目录中的
%% 这个命令\addcontentsline{toc}{chapter}{摘\qquad 要}
%% 使其也加到目录中,且显示为“摘    要”,以下类似
%% 每一章第一页默认是没有页眉的,这里的命令\thispagestyle{main}
%% 使这一页也有页眉,以下类似

本模板由Yongtao Zhou制作,Ming Li二次修改,非官方模板,请选择性使用,最终解释权和所有权归作者所有。仅供个人学习、学术交流使用,可以随意修改,保留作者信息即可$\wedge\_\wedge$。
如果您在使用中有建议和发现任何的bug,欢迎与本人联系。作者邮箱:\newline li-ming96@foxmail.com。
% \par 本项目地址:https:$//$github.com$/$ytZhou$/$JNUMasterThesis

PS:注意做好项目备份,避免造成不必要的损失。由于错误操作造成的损失,作者概不负责。

\bigskip
\noindent {\bfseries 关键词:} 模板;非官方;学术交流;备份


\chapter*{\bfseries \sffamily Abstract}\addcontentsline{toc}{chapter}{Abstract}
\thispagestyle{main}

Artificial intelligence (AI) is a field of computer science focused on creating systems that can perform tasks typically requiring human intelligence. These tasks include problem-solving, learning from data, understanding natural language, recognizing patterns, and decision-making. AI is broadly divided into two types: narrow AI, which is designed for specific tasks like image recognition or language processing, and general AI, which would theoretically perform any intellectual task a human can do. Key methods include machine learning, where systems improve over time from experience, and deep learning, which uses neural networks to process large datasets. AI is widely used across sectors, from healthcare and finance to automotive and entertainment, transforming industries and enhancing efficiency.

\bigskip
\noindent {\bfseries Key Words:} artificial intelligence; machine learning; deep learning
