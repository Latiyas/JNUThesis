%% 算法环境的重置
\floatname{algorithm}{算法}
\renewcommand{\algorithmicrequire}{\textbf{输入:}}
\renewcommand{\algorithmicensure}{\textbf{输出:}}


%% 设置中文字体
%% 加载本地中文字体
\setCJKfamilyfont{song}{simsun.ttc}[AutoFakeSlant,AutoFakeBold={2}]     %% 宋体
\newcommand{\song}{\CJKfamily{song}}                                    %% 宋体
\setCJKfamilyfont{hei}{simhei.ttf}[AutoFakeSlant,AutoFakeBold={2}]      %% 黑体
\newcommand{\hei}{\CJKfamily{hei}}                                      %% 黑体
\setCJKfamilyfont{kai}{simkai.ttf}[AutoFakeSlant,AutoFakeBold={2}]      %% 楷体
\newcommand{\kai}{\CJKfamily{kai}}                                      %% 楷体
\setCJKfamilyfont{li}{SIMLI.TTF}[AutoFakeSlant,AutoFakeBold={2}]        %% 隶书
\newcommand{\li}{\CJKfamily{li}}                                        %% 隶书
% %% 如需加载其他中文字体,可以从网上或者C:\Windows\Fonts复制字体文件到根目录下加载即可。
% \newcommand{\fs}{\CJKfamily{fs}}        %% 仿宋
% \newcommand{\you}{\CJKfamily{you}}      %% 幼圆
\setCJKmainfont{simsun.ttc}[AutoFakeSlant,AutoFakeBold={2}]
\setCJKsansfont{simsun.ttc}[AutoFakeSlant,AutoFakeBold={2}]
%% 设置英文字体
\setmainfont{Times New Roman}
\setsansfont{Arial}
\setmonofont{Times New Roman}


%% 重新设置字体大小,行高 = 字体大小 * 1.2
\newcommand{\wuhao}{\fontsize{10.5pt}{12.6pt}\selectfont}          %% 五号字体
\newcommand{\xiaosi}{\fontsize{12pt}{14.4pt}\selectfont}           %% 小四字体
\newcommand{\sihao}{\fontsize{14pt}{16.8pt}\selectfont}            %% 四号字体
\newcommand{\xiaosan}{\fontsize{15pt}{18pt}\selectfont}            %% 小三字体
\newcommand{\sanhao}{\fontsize{16pt}{19.2pt}\selectfont}           %% 三号字体
\newcommand{\xiaoer}{\fontsize{18pt}{21.6pt}\selectfont}           %% 小二字体
\newcommand{\erhao}{\fontsize{22pt}{26.4pt}\selectfont}            %% 二号字体
%% 汉字字距
\renewcommand{\CJKglue}{\hskip 1pt plus 0.08\baselineskip}
%% 重置默认字体--小四
\renewcommand{\normalsize}{\fontsize{12pt}{14.4pt}\selectfont}


%% 修改行距、段距、缩进距离
%% 行距倍数
\renewcommand{\baselinestretch}{1.625}
%% 段落间距
\setlength{\parskip}{0\baselineskip}
%% 首行缩进
\usepackage{indentfirst}
\setlength{\parindent}{2em}     %% 2em代表首行缩进两个字符


%% latex超链接
\usepackage{hyperref}
%% 引入书签
\hypersetup{hidelinks,
	colorlinks=true,
	allcolors=black,
	pdfstartview=Fit,
	breaklinks=true
}


% 设置章节编号深度
\setcounter{secnumdepth}{3}


%% 用titlesec宏包设置章节标题
%% 用titletoc宏包设置章页眉页脚和目录的格式
\usepackage[indentafter, pagestyles]{titlesec}
\usepackage{titletoc}
%% 设置页眉页脚
\newpagestyle{main}[\wuhao]{
\sethead[][\titleCn][]{}{\pageHeaderTitle}{}
\setfoot{}{\thepage}{}
\setheadrule{0.05pt}
}
%% 设置论文默认页眉页脚格式
\pagestyle{main}
%% 用\titleformat命令设置章标题的格式
\titleformat{\chapter}[hang]{\centering\xiaosan\hei}{\thechapter}{1em}{}
\titleformat{\section}[hang]{\sihao\hei}{\thesection}{0.5em}{}
\titleformat{\subsection}[hang]{\xiaosi\hei}{\thesubsection}{0.5em}{}
\titleformat{\subsubsection}[hang]{\xiaosi\hei}{\thesubsubsection}{0.5em}{}
%% 修改目录中章节标题样式
\ctexset{
    chapter={
    name={}, % 章节标题不加“第”和“章”
    number=\arabic{chapter}, % 使用阿拉伯数字
    }
}
%% 用\titlespacing或\titlespacing*命令设置标题与四周的距离
\titlespacing{\chapter}{0pt}{*0}{*5}
\titlespacing{\section}{0pt}{*1}{*1}
\titlespacing{\subsection}{0pt}{*0.5}{*0.5}
\titlespacing{\subsubsection}{0pt}{*0.5}{*0.5}


%% 重设目录设置
\setcounter{tocdepth}{1}
\titlecontents{chapter}[0pt]{\vspace{0pt}\song\xiaosi}
    {\thecontentslabel\hspace{1em}}{}
    {\hspace{0em}\titlerule*[3pt]{$\cdot$}\contentspage}
\titlecontents{section}[0pt]{\vspace{0pt}\hspace{1em}\song\xiaosi}
    {\thecontentslabel\hspace{1em}}{}
    {\hspace{0em}\titlerule*[3pt]{$\cdot$}\contentspage}
\titlecontents{subsection}[0pt]
    {\vspace{0pt}\hspace{2em}\song\xiaosi}
    {\thecontentslabel\hspace{1em}}{}
    {\hspace{0em}\titlerule*[3pt]{$\cdot$}\contentspage}
%% 重置目录的名称
\renewcommand{\contentsname}{目\qquad 录}
%% 修改图表目录的指引线距离
\makeatletter
\def\@dotsep{1} % 默认是4.5,就是点之间距离为4.5mu。可以改成2
\makeatother


%% 图、表、公式和算法编号重置
\usepackage[labelsep=space]{caption}
%% 设置控制标签与标题之间的间隔形式
\renewcommand{\figurename}{图}
\renewcommand{\tablename}{表}
%% 设置图, 表的编号为短杠形式
\renewcommand\thefigure{\arabic{chapter}-\arabic{figure}}
\renewcommand\thetable{\arabic{chapter}-\arabic{table}}
%% 设置公式、算法的编号为短杠形式
\renewcommand\theequation{\thechapter-\arabic{equation}}
\renewcommand\thealgorithm{\thechapter-\arabic{algorithm}}


%% 将默认文献引用设置为上标出现
\makeatletter
\def\@cite#1#2{\textsuperscript{[{#1\if@tempswa , #2\fi}]}}
\makeatother
%% 重新定义非上标的文献引用
\newcommand{\norcite}[1]{[\citenum{#1}]}


%% 参考文献格式重置,重置参考文献的名称
\def\thebibliography#1{\chapter*{参\ 考\ 文\ 献}
    \addcontentsline{toc}{chapter}{参考文献}
    \thispagestyle{main}
    \list
    {[\arabic{enumi}]}{\settowidth\labelwidth{[#1]}\leftmargin\labelwidth
    \advance\leftmargin\labelsep
    \usecounter{enumi}}
    \def\newblock{\hskip .11em plus .33em minus .07em}
    \sloppy\clubpenalty4000\widowpenalty4000
    \sfcode`\.=1000\relax
}
%% 修改参考文献条目之间的竖直方向上的距离
\patchcmd{\thebibliography}{\leftmargin\labelwidth}{\leftmargin\labelwidth\addtolength\itemsep{-8pt}}{}{}


%% 中文和英文之间加入波浪符“~”,或“ \ ”
%% 例如:中文~abc~中文
%% 中英文混排的时候是用中文标点还是英文标点呢?
%% 这并没有统一的规范。不过比较合理也比较通行的做法是
%% 中文后用中文标点,英文后用英文标点
%% 一般数学文章习惯用全角的实心句点作为中文句号
%% 全角的中文括号看起来不太好看,可以统一使用英文的括号
%% 左括号前面和右括号后面最好加上波浪符“~”
%% 在缩写词后面加上波浪符“~”,如 pp.~100, cf.~something
%% 定义符合中文习惯的定理环境
\newtheoremstyle{mythm}{1.5ex plus 1ex minus .2ex}{1.5ex plus 1ex minus .2ex}{\kai}{\parindent}{\song\bfseries}{}{1em}{}
\theoremstyle{mythm}
\newtheorem{theo}{定理~}[chapter]
\newtheorem{lem}{引理~}
\newtheorem{prop}{命题~}
\newtheorem{cor}{推论~}
\newtheorem{defi}{定义~}
\newtheorem{conj}{猜想~}
\newtheorem{exmp}{例~}
\newtheorem{rem}{注~}
\newtheorem{nota}{记号~}

\newtheoremstyle{specthm}{1.5ex plus 1ex minus .2ex}{1.5ex plus 1ex minus .2ex}{\kai}{\parindent}{\song\bfseries}{}{1em}{\thmnote{#3}}
\theoremstyle{specthm}
%% 旧版本(后期可删)
%% \newtheorem{sthm}[theo]{}
%% 例如:
%% \begin{sthm}[定理~\thetheo~(存在性定理)]
%% 定理内容
%% \end{sthm}

%% 定义符合中文习惯的证明环境
\makeatletter
\renewenvironment{proof}[1][\proofname]{\par
    \pushQED{\qed}%
    \normalfont \topsep6\p@\@plus6\p@ \labelsep1em\relax
    \trivlist
    \item[\hskip\labelsep\indent
        \bfseries #1]\ignorespaces
}{%
    \popQED\endtrivlist\@endpefalse
}
\makeatother

\renewcommand{\proofname}{证明}
%% proof环境会自动在证明最后一行的最右边加上一个证明结束符
%% 默认为空心方块,可以重新定义 \qedsymbol 来修改它
%% 需要注意的是,当证明以一个独立公式结束时
%% 证明结束符会出现在下一行的最右边,而不是在公式的同一行上
%% 这不合乎习惯。这时只要在公式环境内加上 \qedhere 即可


%% 重置列表设置
%% 设置enumerate的距离
\usepackage{enumitem}
\setlist[enumerate]{itemsep=0pt, parsep=0pt, left=24pt}
%% 将enumerate的标签格式设置为“(1)”类型
\setlist[enumerate,1]{label=(\arabic*)}
%% 设置itemize的距离
\setlist[itemize]{itemsep=0pt, parsep=0pt, left=24pt}



%% 重设每章结束后的\cleardoublepage命令(twoside)
\ifisprint
    % 如果不使用该文字,可以不做任何操作
\else
    % 在这里添加你想要使用的文字或代码段
    \let\cleardoublepage\clearpage
\fi


% 自定义标题字体大小
\DeclareCaptionFont{FTfont}{\fontsize{10.5pt}{12.6pt}\selectfont}
% 设置图、表标题的字体大小
\captionsetup[figure]{font=FTfont}
\captionsetup[table]{font=FTfont}