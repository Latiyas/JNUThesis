%%%%%%%%%%%%%%%%%%%%%%%%%%%%%%%%%%%%%%%%%%%%%%%%%%%%%%%%%%%%%%%%%%%%%%%%%%%%
%% Author: Ming Li
%% Time: 2022-7-18
%% Note: Overleaf版本
%%%%%%%%%%%%%%%%%%%%%%%%%%%%%%%%%%%%%%%%%%%%%%%%%%%%%%%%%%%%%%%%%%%%%%%%%%%%
%%%%%%%%%%%%%%%%%%%%%%%%%%%%%%%%%%%%%%%%%%%%%%%%%%%%%%%%%%%%%%%%%%%%%%%%%%%%
%% Author: Yongtao Zhou <y.t.zhou@foxmail.com>                            %%
%% Time: 2015-4-8                                                         %%
%% Note: 请保留原作者Yongtao Zhou信息,非官方模板。                            %%
%%       最终解释权和所有权利归原作者和数据存储与集群计算实验室所有。               %%
%%%%%%%%%%%%%%%%%%%%%%%%%%%%%%%%%%%%%%%%%%%%%%%%%%%%%%%%%%%%%%%%%%%%%%%%%%%%
\PassOptionsToPackage{quiet}{fontspec}      %% 抑制fontspec警告 
\documentclass[a4paper,oneside,12pt]{book}
\usepackage{amsfonts}
\usepackage{amssymb}
\usepackage{amsmath}
\usepackage{amscd}
\usepackage{amsfonts,eucal}
\usepackage{amsthm}
\usepackage{cite}
\usepackage{bm}
\usepackage[UTF8,fontset=none,heading=true]{ctex}
\usepackage{times}
%% 默认算法环境
\usepackage{algorithm}
\usepackage{algorithmic}
% \usepackage{algorithm2e}
% \usepackage[linesnumbered,ruled,vlined,boxed,commentsnumbered]{algorithm2e}
%% 常用包
\usepackage{textcomp, gensymb}
\usepackage{graphicx}
\usepackage{url}
\usepackage{multirow}
\usepackage{subfig}
\usepackage{float}
\usepackage{subfig}
\usepackage{array}
\usepackage{amsmath}            %% equation mutilrow align
\usepackage{amsthm}
\usepackage{makecell}
\usepackage{rotating}
\usepackage{booktabs}
\usepackage{diagbox}
\usepackage{booktabs}
%% 用geometry宏包设置页边距
\usepackage[top=2.5cm,bottom=2.0cm,left=2.5cm,right=2.5cm]{geometry}
%% 用titlesec宏包设置章节标题
%% 用titletoc宏包设置章页眉页脚和目录的格式
\usepackage[indentafter, pagestyles]{titlesec}
\usepackage{titletoc}
\usepackage{etoolbox}
\usepackage{indentfirst}
\setlength{\parindent}{2em}     %% 2em代表首行缩进两个字符
%% latex超链接
\usepackage{hyperref}

%% 引入书签
\hypersetup{hidelinks,
	colorlinks=true,
	allcolors=black,
	pdfstartview=Fit,
	breaklinks=true
}

%% 用\titleformat命令设置章标题的格式
\titleformat{\chapter}[hang]{\centering\xiaosan\hei}{\thechapter}{1em}{}
\titleformat{\section}[hang]{\sihao\hei}{\thesection}{0.5em}{}
\titleformat{\subsection}[hang]{\xiaosi\hei}{\thesubsection}{0.5em}{}

%% 用\titlespacing或\titlespacing*命令设置标题与四周的距离
\titlespacing{\chapter}{0pt}{*0}{*5}
\titlespacing{\section}{0pt}{*1}{*1}
\titlespacing{\subsection}{0pt}{*0.5}{*0.5}

% %% 重置默认字体--小四
% \renewcommand{\normalsize}{\fontsize{12pt}{14.4pt}\selectfont}

%% 行距倍数
% \renewcommand{\baselinestretch}{1.5}
\renewcommand{\baselinestretch}{1.625}
%% 段落间距
% \setlength{\parskip}{0.3\baselineskip}
\setlength{\parskip}{0\baselineskip}
%% 汉字字距
\renewcommand{\CJKglue}{\hskip 1pt plus 0.08\baselineskip}
%% 重新设置字体大小,行高 = 字体大小 * 1.2
\newcommand{\wuhao}{\fontsize{10.5pt}{12.6pt}\selectfont}          %% 五号字体
\newcommand{\xiaosi}{\fontsize{12pt}{14.4pt}\selectfont}           %% 小四字体
\newcommand{\sihao}{\fontsize{14pt}{16.8pt}\selectfont}            %% 四号字体
\newcommand{\xiaosan}{\fontsize{15pt}{18pt}\selectfont}            %% 小三字体
\newcommand{\sanhao}{\fontsize{16pt}{19.2pt}\selectfont}           %% 三号字体
\newcommand{\xiaoer}{\fontsize{18pt}{21.6pt}\selectfont}           %% 小二字体
\newcommand{\erhao}{\fontsize{22pt}{26.4pt}\selectfont}            %% 二号字体

%% 重设目录设置
\setcounter{tocdepth}{1}
\titlecontents{chapter}[0pt]{\vspace{0pt}\song\xiaosi}
    {\thecontentslabel\hspace{1em}}{}
    {\hspace{0em}\titlerule*[3pt]{$\cdot$}\contentspage}
\titlecontents{section}[0pt]{\vspace{0pt}\hspace{1em}\song\xiaosi}
    {\thecontentslabel\hspace{1em}}{}
    {\hspace{0em}\titlerule*[3pt]{$\cdot$}\contentspage}
\titlecontents{subsection}[0pt]{\vspace{0pt}\hspace{2em}\song\xiaosi}
    {\thecontentslabel\hspace{1em}}{}
    {\hspace{0em}\titlerule*[3pt]{$\cdot$}\contentspage}

% %% 设置字体大小
% \newcommand{\wuhao}{\fontsize{10.5pt}{12.6pt}\selectfont}        %% 五号字体
% \newcommand{\xiaosi}{\fontsize{12pt}{18pt}\selectfont}           %% 小四字体
% \newcommand{\sihao}{\fontsize{14pt}{21pt}\selectfont}            %% 四号字体
% \newcommand{\xiaosan}{\fontsize{15pt}{22.5pt}\selectfont}        %% 小三字体
% \newcommand{\sanhao}{\fontsize{16pt}{24pt}\selectfont}           %% 三号字体
% \newcommand{\xiaoer}{\fontsize{18pt}{27pt}\selectfont}           %% 小二字体
% \newcommand{\erhao}{\fontsize{22pt}{30pt}\selectfont}            %% 二号字体
% 
% %% 目录设置
% \setcounter{tocdepth}{1}
% \titlecontents{chapter}[0pt]{\vspace{-15pt}\song\xiaosi}
%     {\thecontentslabel\hspace{1em}}{}
%     {\hspace{0em}\titlerule*[3pt]{$\cdot$}\contentspage}
% \titlecontents{section}[0pt]{\vspace{-15pt}\hspace{1em}\song\xiaosi}
%     {\thecontentslabel\hspace{1em}}{}
%     {\hspace{0em}\titlerule*[3pt]{$\cdot$}\contentspage}
% \titlecontents{subsection}[4em]{\vspace{-15pt}\song\xiaosi}
%     {\thecontentslabel\hspace{1em}}{}
%     {\hspace{0em}\titlerule*[3pt]{$\cdot$}\contentspage}

%% 设置中文字体
%% 加载本地中文字体
\setCJKfamilyfont{song}{simsun.ttc}[AutoFakeSlant,AutoFakeBold={2}]     %% 宋体
\newcommand{\song}{\CJKfamily{song}}                                    %% 宋体
\setCJKfamilyfont{hei}{simhei.ttf}[AutoFakeSlant,AutoFakeBold={2}]      %% 黑体
\newcommand{\hei}{\CJKfamily{hei}}                                      %% 黑体
\setCJKfamilyfont{kai}{simkai.ttf}[AutoFakeSlant,AutoFakeBold={2}]      %% 楷体
\newcommand{\kai}{\CJKfamily{kai}}                                      %% 楷体
\setCJKfamilyfont{li}{SIMLI.TTF}[AutoFakeSlant,AutoFakeBold={2}]        %% 隶书
\newcommand{\li}{\CJKfamily{li}}                                        %% 隶书

%%% 如需加载其他中文字体,可以从网上或者C:\Windows\Fonts复制字体文件到根目录下加载即可。
%% \newcommand{\fs}{\CJKfamily{fs}}        %% 仿宋
%% \newcommand{\you}{\CJKfamily{you}}      %% 幼圆
\setCJKmainfont{simsun.ttc}[AutoFakeSlant,AutoFakeBold={2}]

%% 设置英文字体
\RequirePackage{xltxtra}                %% \XeTeX Logo
\setmainfont{Times New Roman}
\setsansfont{Arial}
\setmonofont{Times New Roman}

%% 设置页眉页脚
\newpagestyle{main}[\wuhao]{
\sethead{}{\pageHeaderTitle}{}
\setfoot{}{\thepage}{} \setheadrule{0.05pt}}\pagestyle{main}

%% 重置目录的名称
\renewcommand{\contentsname}{目\qquad 录}

%% 算法环境的重置
\floatname{algorithm}{算法}
\renewcommand{\algorithmicrequire}{\textbf{输入:}}
\renewcommand{\algorithmicensure}{\textbf{输出:}}
% \renewcommand{\algorithmcfname}{算法}
% \SetKwInput{KwData}{输入}
% \SetKwInput{KwResult}{输出}

%% 图表重置
%% 设置公式的编号为短杠形式
\renewcommand\theequation{%
\thechapter-\arabic{equation}}
%% 设置图, 表的编号为短杠形式
\renewcommand\thefigure{\arabic{chapter}-\arabic{figure}}
\renewcommand\thetable{\arabic{chapter}-\arabic{table}}
%% 设置控制标签与标题之间的间隔形式
\renewcommand{\figurename}{图}
\renewcommand{\tablename}{表}
\usepackage[labelsep=space]{caption}

%% 中文和英文之间加入波浪符“~”,或“ \ ”
%% 例如:中文~abc~中文
%% 中英文混排的时候是用中文标点还是英文标点呢?
%% 这并没有统一的规范。不过比较合理也比较通行的做法是
%% 中文后用中文标点,英文后用英文标点
%% 一般数学文章习惯用全角的实心句点作为中文句号
%% 全角的中文括号看起来不太好看,可以统一使用英文的括号
%% 左括号前面和右括号后面最好加上波浪符“~”
%% 在缩写词后面加上波浪符“~”,如 pp.~100, cf.~something
%% 定义符合中文习惯的定理环境
\newtheoremstyle{mythm}{1.5ex plus 1ex minus .2ex}{1.5ex plus 1ex minus .2ex}{\kai}{\parindent}{\song\bfseries}{}{1em}{}
\theoremstyle{mythm}
\newtheorem{thm}{定理~}[chapter]
\newtheorem{lem}{引理~}
\newtheorem{prop}{命题~}
\newtheorem{cor}{推论~}
\newtheorem{defn}{定义~}
\newtheorem{conj}{猜想~}
\newtheorem{exmp}{例~}
\newtheorem{rem}{注~}
\newtheorem{notation}{记号~}

\newtheoremstyle{specthm}{1.5ex plus 1ex minus .2ex}{1.5ex plus 1ex minus .2ex}{\kai}{\parindent}{\song\bfseries}{}{1em}{\thmnote{#3}}
\theoremstyle{specthm}
\newtheorem{sthm}[thm]{}
%% 例如:
%% \begin{sthm}[定理~\thethm~(存在性定理)]
%% 定理内容
%% \end{sthm}
%% 定义符合中文习惯的证明环境
\makeatletter
\renewenvironment{proof}[1][\proofname]{\par
    \pushQED{\qed}%
    \normalfont \topsep6\p@\@plus6\p@ \labelsep1em\relax
    \trivlist
    \item[\hskip\labelsep\indent
        \bfseries #1]\ignorespaces
}{%
    \popQED\endtrivlist\@endpefalse
}
\makeatother


%% 将默认文献引用设置为上标出现
\makeatletter
\def\@cite#1#2{\textsuperscript{[{#1\if@tempswa , #2\fi}]}}
\makeatother
%% 重新定义非上标的文献引用
\newcommand{\norcite}[1]{\scalebox{1.5}[1.5]{\raisebox{-0.75ex}{\cite{#1}}}}


%% 修改图表目录的指引线距离
\makeatletter
\def\@dotsep{1} % 默认是4.5,就是点之间距离为4.5mu。可以改成2
\makeatother

\renewcommand{\proofname}{证明}
%% proof环境会自动在证明最后一行的最右边加上一个证明结束符
%% 默认为空心方块,可以重新定义 \qedsymbol 来修改它
%% 需要注意的是,当证明以一个独立公式结束时
%% 证明结束符会出现在下一行的最右边,而不是在公式的同一行上
%% 这不合乎习惯。这时只要在公式环境内加上 \qedhere 即可

% Declarative information
%% 声明页信息来源,需要将个人信息进行填充
%%-------------------这里定义论文相关信息--------------------
%% 作者名字
\newcommand{\authorName}{作者姓名(若是同等学力人员请注明“同等学力申请”,若是港澳台侨及海外留学生请注明申请人生源地)}
%% 论文题目
\newcommand{\titleCn}{暨南大学硕士学位论文Overleaf模板v1.0}
\newcommand{\titleEn}{论文英文题目}
%% 导师及其职称
\newcommand{\tutor}{导师姓名\ 导师学位\ 导师职称}
%% 学科、专业名称
\newcommand{\disciplineName}{学科\ 专业}
%% 学位类型
\newcommand{\degreeType}{(学术学位/专业学位)}
%% 论文提交时间
\newcommand{\commitDate}{2024\ 年\ 6\ 月}
%% 论文答辩时间
\newcommand{\replyDate}{2024\ 年\ 6\ 月}
%% 答辩委员会主席
\newcommand{\programCoChairs}{委员会主席\ 职称}
%% 论文评阅人
\newcommand{\Reviewer}{评阅人名字\ 职称}
%% 学位论文授予单位日期
\newcommand{\grantDate}{2024\ 年\ 6\ 月}
\newcommand{\schoolName}{暨南大学}
\newcommand{\pageHeaderTitle}{暨南大学硕士学位论文}

% Added Library
\input{chapter/Lib}


\begin{document}
%% 参考文献格式重置,重置参考文献的名称
\def\thebibliography#1{\chapter*{参\ 考\ 文\ 献}
    \addcontentsline{toc}{chapter}{参考文献}
    \thispagestyle{main}
    \list
    {[\arabic{enumi}]}{\settowidth\labelwidth{[#1]}\leftmargin\labelwidth
    \advance\leftmargin\labelsep
    \usecounter{enumi}}
    \def\newblock{\hskip .11em plus .33em minus .07em}
    \sloppy\clubpenalty4000\widowpenalty4000
    \sfcode`\.=1000\relax
}

%% 修改参考文献条目之间的竖直方向上的距离
\patchcmd{\thebibliography}{\leftmargin\labelwidth}{\leftmargin\labelwidth\addtolength\itemsep{-8pt}}{}{}

%% 前言
%%-------------------这是标题页--------------------
\begin{titlepage}
\begin{flushleft}
    \zihao{-1}\li\textbf{暨南大学硕士学位论文}
\end{flushleft}
\begin{flushleft}
\vspace{25pt}
\xiaosan{
题名: \titleCn \\
Title: \titleEn \\[25pt]
作者姓名:\authorName \\[25pt]
指导教师姓名及学位、职称:\tutor \\[25pt]
学科、专业名称:\disciplineName \\[25pt]
学位类型:\degreeType \\[25pt]
论文提交时间:\commitDate \\[25pt]
论文答辩时间:\replyDate \\[25pt]
答辩委员会主席:\programCoChairs \\[25pt]
论文评阅人:\Reviewer \\[25pt]
学位授予单位日期:\grantDate \\[25pt]
}
\end{flushleft}
\end{titlepage}

\begin{titlepage}
\begin{center}
    \zihao{2}\song\textbf{独~创~性~声~明}
\end{center}
\par 本人声明所呈交的学位论文是本人在导师指导下进行的研究工作及取得的研究成果。除了文中特别加以标注和致谢的地方外,论文中不包含其他人已经发表或撰写过的研究成果,也不包含为获得\kai{\zihao{3}\textbf{\underline{~~~~\schoolName~~~~}}}\song 或其他教育机构的学位或证书而使用过的材料。与我一同工作的同志对本研究所做的任何贡献均已在论文中作了明确的说明并表示谢意。\\[25pt]
\par 学位论文作者签名:\hspace{115pt}签字日期:~~~~~~年~~~~~~月~~~~~~日\\[25pt]
\begin{center}
    \song\zihao{2}\textbf{学位论文版权使用授权书}
\end{center}
\par 本学位论文作者完全了解\kai{\zihao{3}\textbf{\underline{~~~~~\schoolName~~~~~}}}\song 有关保留、使用学位论文的规定,有权保留并向国家有关部门或机构送交论文的复印件和磁盘,允许论文被查阅和借阅。本人授权\kai{\zihao{3}\textbf{\underline{~~~~\schoolName~~~~}}}\song 可以将学位论文的全部或部分内容编入有关数据库进行检索,可以采用影印、缩印或扫描等复制手段保存、汇编学位论文。
\par (保密的学位论文在解密后适用本授权书)\\[25pt]
\begin{flushleft}
学位论文作者签名:\hspace{135pt}导师签名:\\
签字日期:~~~~~~年~~~~~~月~~~~~~日\hspace{95pt}签字日期:~~~~~~年~~~~~~月~~~~~~日\\
学位论文作者毕业后去向:\\
工作单位:\hspace{185pt}电话:\\
通讯地址:\hspace{185pt}邮编:\\
\end{flushleft}
\end{titlepage}

%% 摘要
\frontmatter \pagenumbering{Roman}  %% 采用大写罗马数字,缺省时为小写的

\chapter*{摘\texorpdfstring{\qquad}{} 要}\addcontentsline{toc}{chapter}{摘要}
\thispagestyle{main}
%% 加星号*的章节默认是不加入目录中的
%% 这个命令\addcontentsline{toc}{chapter}{摘\qquad 要}
%% 使其也加到目录中,且显示为“摘    要”,以下类似
%% 每一章第一页默认是没有页眉的,这里的命令\thispagestyle{main}
%% 使这一页也有页眉,以下类似

本模板由Yongtao Zhou制作,Ming Li二次修改,非官方模板,请选择性使用,最终解释权和所有权归作者所有。仅供个人学习、学术交流使用,可以随意修改,保留作者信息即可$\wedge\_\wedge$。
如果您在使用中有建议和发现任何的bug,欢迎与本人联系。作者邮箱:\newline li-ming96@foxmail.com。
% \par 本项目地址:https:$//$github.com$/$ytZhou$/$JNUMasterThesis

PS:注意做好项目备份,避免造成不必要的损失。由于错误操作造成的损失,作者概不负责。

\bigskip
\noindent {\bfseries 关键词:} 模板;非官方;学术交流;备份


\chapter*{\bfseries \sffamily Abstract}\addcontentsline{toc}{chapter}{Abstract}
\thispagestyle{main}

Artificial intelligence (AI) is a field of computer science focused on creating systems that can perform tasks typically requiring human intelligence. These tasks include problem-solving, learning from data, understanding natural language, recognizing patterns, and decision-making. AI is broadly divided into two types: narrow AI, which is designed for specific tasks like image recognition or language processing, and general AI, which would theoretically perform any intellectual task a human can do. Key methods include machine learning, where systems improve over time from experience, and deep learning, which uses neural networks to process large datasets. AI is widely used across sectors, from healthcare and finance to automotive and entertainment, transforming industries and enhancing efficiency.

\bigskip
\noindent {\bfseries Key Words:} artificial intelligence; machine learning; deep learning


%% 生成目录
\tableofcontents
\thispagestyle{main}

%% 重置图/表目录(取消不同章之间的间隔)
\begingroup
	\renewcommand*{\addvspace}[1]{}
	\newcommand{\loflabel}{图} 
	\renewcommand{\numberline}[1]{\loflabel~#1\hspace*{1em}}
    \renewcommand{\listfigurename}{图目录}
	\listoffigures
	
	\newcommand{\lotlabel}{表}
	\renewcommand{\numberline}[1]{\lotlabel~#1\hspace*{1em}}
    \renewcommand{\listtablename}{表目录}
	\listoftables
\endgroup
% %%图目录
% \listoffigures
% \thispagestyle{main}
% %%表格目录.
% \listoftables
% \thispagestyle{main}

%% 正文
\mainmatter
%% 设置章节格式(一倍行距为字体大小 × 1.297(1.3) × 1.25)
\ctexset{chapter={
        name = {},
        number = {\arabic{chapter}},
        format = {\bfseries \sffamily \hei \centering \zihao{-3}},
        beforeskip = 15pt,
        afterskip = 25pt,
        fixskip = true,
        aftername = ~~~~,
    }
}
%% 设置一级章节格式
\ctexset{section={
        format = {\bfseries \sffamily \hei \raggedright \zihao{4}},
        beforeskip = 25pt,
        afterskip = 20pt,
        fixskip = true,
    }
}
%% 设置二级标题格式 
\ctexset{subsection={
        format = {\bfseries \sffamily \hei \raggedright \zihao{-4}},
        beforeskip = 20pt,
        afterskip = 20pt,
        fixskip = true,
    }
}
%% 设置三节标题格式
\ctexset{subsubsection={
        format = {\bfseries \sffamily \hei \raggedright \zihao{-4}},
        beforeskip = 20pt,
        afterskip = 20pt,
        fixskip = true,
    }
}

%% 论文正文部分,在chaps.tex内编辑
%% 引入正文章节
\include{chapter/chap1}
\chapter{重复数据删除相关技术综述}\thispagestyle{main}
\section{伪代码实例}
\begin{algorithm}
\caption{Algorithm 1}
\begin{algorithmic}[1] %[1] 能够显示行号
    \REQUIRE $n \geq 1$                  %输入条件
    \ENSURE $Sum = 1 + \cdots + n$       %输出
    \STATE $Sum \leftarrow 0$            %\STATE 命名演示
    \IF {$n < 1$}                        %条件语句
        \PRINT {Input Error}                 %打印语句
    \ELSE
        \FOR {$i = 0$ to n}          %FOR循环结构
            \STATE $Sum = Sum + i$\\
            \STATE $i = i + 1$
        \ENDFOR
    \ENDIF
    \RETURN Sum
\end{algorithmic}
\end{algorithm}
% \begin{algorithm}
% \DontPrintSemicolon
% \KwData{$G=(X,U)$ such that $G^{tc}$ is an order.}
% \KwResult{$G’=(X,V)$ with $V\subseteq U$ such that $G’^{tc}$ is an
% interval order.}
% \Begin{
%     $V \longleftarrow U$\;
%     $S \longleftarrow \emptyset$\;
%     \nl\While{$S \neq \emptyset$}{\label{InRes1}
%     \nlset{REM} remove $x$ from the list of $T$ of maximal index\;\label{InResR}
%     \lnl{InRes2}\While{$|S \cap ImSucc(x)| \neq |S|$}{
%     \For{$ y \in S-ImSucc(x)$}{
%         \{ remove from $V$ all the arcs $zy$ : \}\;
%     \For{$z \in ImPred(y) \cap Min$}{
%         remove the arc $zy$ from $V$\;
%         $NbSuccInS(z) \longleftarrow NbSuccInS(z) - 1$\;
%         move $z$ in $T$ to the list preceding its present list\;
%         \{i.e. If $z \in T[k]$, move $z$ from $T[k]$ to
%         $T[k-1]$\}\;
%     }
%     $NbPredInMin(y) \longleftarrow 0$\;
%     $NbPredNotInMin(y) \longleftarrow 0$\;
%     $S \longleftarrow S - \{y\}$\;
%     $AppendToMin(y)$\;
%     }
%     }
%     $RemoveFromMin(x)$\;
%     }
% }
% \caption{IntervalRestriction\label{IR}}
% \end{algorithm}


%% 结论
\chapter{结论}\thispagestyle{main}

结论整篇论文。


%% 参考文献引入
%% 暨大参考文献设置
% \bibliographystyle{thubib}
% 参考文献设置-GB
\bibliographystyle{gbt7714-2015}
\bibliography{bib/refs}

%% 攻读学位期间发表的论文
\include{chapter/papers}

%% 致谢
\chapter*{致\texorpdfstring{\qquad}{} 谢}\addcontentsline{toc}{chapter}{致谢}
\thispagestyle{main}
\par 感谢622实验室及其他同学的帮助和支持。

%% 加入\newpage能使最后一页也有页眉
\newpage
\end{document}